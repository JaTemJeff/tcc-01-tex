%---------- Primeiro Capitulo ----------
\chapter{Introdu\c{c}\~ao}

Afetando cerca de 1\% da popula\c{c}\~ao mundial e codificada na \sigla	{CID-10} {Classifica\c{c}\~ao Internacional de Doen\c{c}as} com os caracteres F98.5, a gagueira \'e cientificamente considerada como dist\'urbio ou transtorno de flu\^encia da fala \cite{Merlo2013}. Ou seja, \'e um dist\'urbio neurol\'ogico e involunt\'ario, caracterizado por interrup\c{c}\~oes ou prolongamentos, aud\'iveis ou n\~ao de sons e s\'ilabas \cite{Buechel2004}.

A \sigla{RAA}{retroalimenta\c{c}\~ao auditiva atrasada} \'e um m\'etodo de tratamento da gagueira, que utiliza-se de duas grandezas, a frequ\^encia e o atraso (\textit{delay}), para proporcionar o efeito coro, causado quando uma pessoa que gagueja, fala ou l\^e ao mesmo tempo que outra pessoa, ou seja, faz com que  a pessoa que gagueja ou\c{c}a suas pr\'oprias palavras com um certo atraso tendo a sensa\c{c}\~ao de que est\'a falando junto com outros \cite{Udemo2008}.

Um aparelho tecnol\'ogico que oferece o RAA como funcionalidade \'e o \textit{SpeechEasy} da Microsom, que se assemelha muito em sua apar\^encia, com um aparelho para deficientes auditivos. Segundo a Microsom, o \textit{SpeechEasy} tem efici\^encia em 75\% das pessoas que o utilizam e cerca de 80\% dos clientes que adquiriram o produto, est\~ao satisfeitos com o resultado \cite{Microson2015}. Uma pesquisa realizada com 31 participantes que possuem gagueira, registrou resultados parecidos, apresentando melhorias de cerca de 79\% na leitura e 61\% na fala auto-expressiva dos participantes com a utiliza\c{c}\~ao do aparelho \cite{Andrade2008}. 

Existem aplicativos que exercem a funcionalidade de simular o efeito coro, tanto para dispositivos m\'oveis como para computadores de mesa e notebooks. Para realizar a simula\c{c}\~ao do efeito coro de maneira adequada \'e recomendado utilizar um aparelho de reprodu\c{c}\~ao que contenha microfone, podendo ser um fone de ouvido convencional ou um \textit{headset} de sua prefer\^encia.

 
\section{Problema}

Custando aproximadamente 10 mil reais e podendo ser adquirido somente sob consulta com uma fonoaudi\'ologa especializada, o \textit{SpeechEasy} acaba se tornando uma op\c{c}\~ao restrita para pessoas com poucas condi\c{c}\~oes financeiras. Segundo o \sigla{IBF}{Instituto Brasileiro de Flu\^encia} a Microson est\'a em contato com o Minist\'erio de Sa\'ude para que o aparelho seja disponibilizado pelo SUS, por\'em enquanto isso n\~ao ocorre, sua disponibilidade \'e limitada para quem tem condi\c{c}\~oes de investir cerca de 10 sal\'arios m\'inimos neste produto.

Ao contr\'ario da plataforma \textit{Window}s onde existe a ferramenta "Mais Flu\^encia"\ que oferece a funcionalidade de simular o efeito coro, gratuitamente e sem limita\c{c}\~oes. Para plataforma Android encontra-se diversos aplicativos que sequer conseguem atender essa funcionalidade e quando atendem existem limita\c{c}\~oes em suas vers\~oes gratuitas.

Existe uma grande dificuldade em encontrar uma ferramenta para dispositivos m\'oveis que realmente atenda a funcionalidade de simular o efeito coro, que seja fornecida gratuitamente sem restric\~oes de utiliza\c{c}\~ao. 

\section{Justificativa}

O problema no Brasil \'e que o SpeechEasy n\~ao pode ser adquirido pela maioria das pessoas que necessitam, devido ao seu valor. Existem outras solu\c{c}\~oes como aplicativos mobile que tentam fazer o mesmo papel por\'em utilizando um fone de ouvido \textit{bluetooth} ou qualquer outro tipo de dispositivo de reprodu\c{c}\~ao. 

\'E muito dif\'icil encontrar aplicativos que disponibilizam essa funcionalidade de maneira eficiente e gratuita, outros aplicativos que trazem a fun\c{c}\~ao de simular o efeito coro de forma simples e bem superficial n\~ao s\~ao gratuitos, o que torna dif\'icil obter resultados satisfat\'orios. 

Realizar apresenta\c{c}\~oes, semin\'arios ou quaisquer atividades que necessitam de atividade vocal na universidade, para pessoas com gagueira \'e uma tarefa bem dif\'icil, pois al\'em da dificuldade de demonstrar conhecimento sobre o assunto exposto, existe tamb\'em a dificuldade para se expressar de maneira fluente. 

O problema na universidade \'e que n\~ao existem mecanismos que auxiliam esses alunos a lidarem com essas situa\c{c}\~oes, o que acaba muitas vezes fazendo com que a pessoa que tem gagueira desista de realizar determinadas atividades pela dificuldade de comunica\c{c}\~ao.


\section{Objetivos}

\subsection{Objetivo Geral}

Desenvolver um aplicativo para dispositivos m\'oveis com sistema operacional Android, gratuito e funcional, que atenda o requisito principal de simular o efeito coro. 

\subsection{Objetivos Espec\'ificos}

\begin{itemize}
	\item Inclus\~ao social de pessoas com gagueira em atividades que exijam comunica\c{c}\~ao oral, com a utiliza\c{c}\~ao do aplicativo.
	\item Acessibilidade para pessoas com poucas condi\c{c}\~oes financeiras para adquirir o \textit{SpeechEasy}, tornando o aplicativo uma alternativa gratuita que exerce o mesmo papel.
	\item Disponibilizar uma op\c{c}\~ao gratuita para fonoaudi\'ologos e profissionais da \'area de aux\'ilio ao tratamento de pessoas com gagueira, utilizando o RAA.
\end{itemize} 

\section{Organiza\c{c}\~ao do Texto}

O documento est� organizado em cap\'itulos, dividos em:

\begin{itemize}
	\item Cap\'itulo 2: apresenta a fundamenta\c{c}\~ao te\'orica dando \^enfase nos trabalhos relacionados, mostrando aplicativos que tenham similaridades com a ferramenta desenvolvida no presente trabalho. 
	
	\item Capitulo 3: apresenta a proposta, citando as tecnologias e ferramentas utilizadas, especificando qual o m\'etodo seguido para o desenvolvimento, a an\'alise e desenvolvimento, onde apresenta-se os requisitos do sistema, os diagramas e prot\'otipos de tela, e por fim o cronograma a ser seguido.
	
	\item Refer\^encias: apresenta as refer\^encias utilizadas. 
	 
\end{itemize}
