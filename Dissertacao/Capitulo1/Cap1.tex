%---------- Primeiro Capitulo ----------
\chapter{Introdu\c{c}\~ao}

Codificada na Classifica\c{c}\~ao Internacional de Doen\c{c}as (CID-10) com os caracteres F98.5, a gagueira \'e cientificamente considerada como dist\'urbio ou transtorno de flu\^encia da fala \cite{Merlo2018}. Ou seja, \'e um dist\'urbio neurol\'ogico e involunt\'ario, caracterizado por interrup\c{c}\~oes ou prolongamentos, aud\'iveis ou n\~ao de sons e s\'ilabas \cite{Buechel2004}.

A retroalimenta\c{c}\~ao auditiva atrasada (RAA) \'e um m\'etodo de tratamento da gagueira, que utiliza-se de duas grandezas, a frequ\^encia e o atraso (delay), para proporcionar o efeito coro, causado quando uma pessoa que gagueja, fala ou l\^e ao mesmo tempo que outra pessoa, proporcionando melhorias significativas na flu\^encia.
Um aparelho tecnol\'ogico que oferece o RAA como funcionalidade \'e o SpeechEasy da Microsom, que se assemelha muito em sua apar\^encia, com um aparelho para deficientes auditivos. Segundo a Microsom, o SpeechEasy tem eficiência em 75 porcento das pessoas que o utilizam e cerca de 80 porcento dos clientes que adquiriram o produto, est\~ao satisfeitos com o resultado.

\section{Motiva\c{c}\~ao}

Uma das principais vantagens do uso do estilo de formata\c{c}\~ao {\ttfamily utfprcptex.cls} para \LaTeX\ \'e a formata\c{c}\~ao \textit{autom\'atica} dos elementos que comp\~oem um documento acad\^emico, tais como capa, folha de rosto, dedicat\'oria, agradecimentos, ep\'igrafe, resumo, abstract, listas de figuras, tabelas, siglas e s\'imbolos, sum\'ario, cap\'itulos, refer\^encias, etc. Outras grandes vantagens do uso do \LaTeX\ para formata\c{c}\~ao de documentos acad\^emicos dizem respeito \`a facilidade de gerenciamento de refer\^encias cruzadas e bibliogr\'aficas, al\'em da formata\c{c}\~ao~-- inclusive de equa\c{c}\~oes  matem\'aticas~-- correta e esteticamente perfeita.

\section{Objetivos}

\subsection{Objetivo Geral}

Prover um modelo de formata\c{c}\~ao \LaTeX\ que atenda \`as Normas para Elabora\c{c}\~ao de Trabalhos Acad\^emicos da UTFPR~\cite{UTFPR2008}.

\subsection{Objetivos Espec\'ificos}

\begin{itemize}
	\item Obter documentos acad\^emicos automaticamente formatados com corre\c{c}\~ao e perfei\c{c}\~ao est\'etica.
	\item Desonerar autores da tediosa tarefa de formatar documentos acad\^emicos, permitindo sua concentra\c{c}\~ao no conte\'udo do mesmo.
	\item Desonerar orientadores e examinadores da tediosa tarefa de conferir a formata\c{c}\~ao de documentos acad\^emicos, permitindo sua concentra\c{c}\~ao no conte\'udo do mesmo.
\end{itemize} 