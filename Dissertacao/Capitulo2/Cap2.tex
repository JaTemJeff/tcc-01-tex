%%%%%%%%%%%%%%%%%%%%%%%%%%%%%%%%%%%%%%%%
%---------- Segundo Capitulo ----------
\chapter{Fundamenta\c{c}\~ao Te\'orica}
\label{chap:desenv}
%%%%%%%%%%%%%%%%%%%%%%%%%%%%%%%%%%%%%%%%

Com o objetivo de simular o efeito coro, al�m do \textit{SpeechEasy} que integra \textit{hardware} e \textit{software}, existem algumas ferramentas que trabalham somente com \textit{software} e que exercem essa fun\c{c}\~ao juntamente com algum dispositivo de reprodu\c{c}\~ao de a\'udio que contenha microfone. 

Para computadores de mesa e notebooks com sistema operacional Windows, existe a ferramenta "Software Mais Flu�ncia Win DAF/FAF Software", desenvolvida em 2009 pelo Henrique Confessor, \'e \textit{freeware} podendo ser distribu\'ida e utilizada livremente. Disponibilizada gratuitamente para \textit{download} no site da "Abra Gagueira" \cite{Confessor2009}. 

Para dispositivos m\'oveis com sistema operacional Android ou IOS existe o \textit{DAF Assistant} que tem uma vers\~ao gratuita, por\'em com limite de tempo para sua utiliza\c{c}\~ao, j\'a sua vers\~ao paga que n\~ao possui essa restri\c{c}\~ao, custa aproximadamente 13 reais na \textit{Play Store} e 33 reais no \textit{Itunes}, variando de acordo com pre�o do dollar \cite{Artefact2012}.

Com o intuito de fornecer o \sigla{FAA}{\textit{Feedback} Auditivo Atrasado}, existe o aplicativo "Terapia para a gagueira - FAA", que \'e gratuito e traz informa\c{c}\~oes interessantes sobre o tratamento da gagueira, como dicas de como utilizar o aplicativo e informa��es adicionais sobre tratamentos que melhoram a flu\^encia da fala. Lembrando que diferente da \sigla{RAA}{Retroalimenta\c{c}\~ao Auditiva Atrasada}, o FAA trabalha apenas com o atraso na reprodu��o da voz, n\~ao alterando a frequ\^encia com que a voz \'e reproduzida \cite{Age2017}. 







