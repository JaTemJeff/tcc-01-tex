%---------- Terceiro Capitulo ----------
\chapter{Proposta}

A proposta desse projeto \'e desenvolver um aplicativo para dispositivos m\'oveis com sistema operacional Android, tendo como fun\c{c}\~ao principal o \sigla{RAA}{Retroalimenta\c{c}\~ao Auditiva Atrasada}, ou seja, um aplicativo que consiga reproduzir a voz do usu\'ario simultaneamente com um pequeno atraso configur\'avel, num tom diferente tamb\'em configur\'avel. 

A finalidade dessas configura\c{c}\~oes que devem ser adaptadas para cada indiv\'iduo \'e simular o efeito coro, que nada mais \'e do que um efeito causado quando uma pessoa que possui gagueira, fala ou l\^e ao mesmo tempo que outra, trazendo melhorias significativas na fala \cite{Udemo2008}.

\section{Tecnologias e Ferramentas}
\begin{itemize}
	
	\item Java: utiliza-se como linguagem de programa\c{c}\~ao.
	
	\item Android Studio: utiliza-se como ambiente de desenvolvimento 		(\sigla{IDE}{Ambiente de Desenvolvimento Integrado}).

	\item Github: utiliza-se como reposit\'orio de armazenamento e controle de vers\~oes.

	\item UML: utiliza-se como linguagem-padr\~ao para a elabora\c{c}\~ao da estrutura de projetos de \textit{software}.

	\item Astah: utiliza-se como ferramenta de modelagem \sigla{UML}{Linguagem Unificada de Modelagem}. 

\end{itemize}

\section{M\'etodo}

\section{An'alise e Desenvolvimento}

\section{Cronograma}

\begin{table}[!htb]
	 \caption{Cronograma de atividades}
	 \label{tab:cronograma}
	 \begin{center}
		  \begin{tabular}{l||c|c|c|c|c|c|c|c}
			    \hline
			    \multicolumn{9}{c}{2018} \\ \hline \hline
			    \multicolumn{1}{c||}{Fase} & M\^es 1     & M\^es 2     & M\^es 3     & M\^es 4   & M\^es 5 & M\^es 6 & M\^es 7 & M\^es 8\\ \hline
			    Levantamento de requisitos    & $\bullet$ &           &          &         &  &  &  & \\
			    An\'alise de requisitos    & $\bullet$          &  & &         &  &  &  &\\
			    Projeto   &           &  $\bullet$         & & &  &  &  &\\
			    Implementa\c{c}\~ao    &           &          &   $\bullet$       &  $\bullet$        &  $\bullet$  & $\bullet$ &  &\\
			    Testes    &           &           & &$\bullet$ & $\bullet$ & $\bullet$ &$\bullet$  &\\
			    Implanta\c{c}\~ao    &           &           & & &  &  &  &$\bullet$\\
			    \hline
			  \end{tabular}
		 \vspace{8pt} %%%% Deve ser acrescentado para que haja espaço entre o final da tabela e a fonte.
		  \fonte{Fonte Modelo.}
		 \end{center}
\end{table}